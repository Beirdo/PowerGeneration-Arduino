% !TEX TS-program = pdflatex
% !TEX encoding = UTF-8 Unicode

% This is a simple template for a LaTeX document using the "article" class.
% See "book", "report", "letter" for other types of document.

\documentclass[11pt]{article} % use larger type; default would be 10pt

\usepackage[utf8]{inputenc} % set input encoding (not needed with XeLaTeX)

%%% Examples of Article customizations
% These packages are optional, depending whether you want the features they provide.
% See the LaTeX Companion or other references for full information.

%%% PAGE DIMENSIONS
\usepackage{geometry} % to change the page dimensions
\geometry{letterpaper} % or letterpaper (US) or a5paper or....
% \geometry{margin=2in} % for example, change the margins to 2 inches all round
% \geometry{landscape} % set up the page for landscape
%   read geometry.pdf for detailed page layout information

\usepackage{graphicx} % support the \includegraphics command and options
\usepackage{mathtools}

% \usepackage[parfill]{parskip} % Activate to begin paragraphs with an empty line rather than an indent

%%% PACKAGES
\usepackage{booktabs} % for much better looking tables
\usepackage{array} % for better arrays (eg matrices) in maths
\usepackage{paralist} % very flexible & customisable lists (eg. enumerate/itemize, etc.)
\usepackage{verbatim} % adds environment for commenting out blocks of text & for better verbatim
\usepackage{subfig} % make it possible to include more than one captioned figure/table in a single float
% These packages are all incorporated in the memoir class to one degree or another...

%%% HEADERS & FOOTERS
\usepackage{fancyhdr} % This should be set AFTER setting up the page geometry
\pagestyle{fancy} % options: empty , plain , fancy
\renewcommand{\headrulewidth}{0pt} % customise the layout...
\lhead{}\chead{}\rhead{}
\lfoot{}\cfoot{\thepage}\rfoot{}

%%% SECTION TITLE APPEARANCE
\usepackage{sectsty}
\allsectionsfont{\sffamily\mdseries\upshape} % (See the fntguide.pdf for font help)
% (This matches ConTeXt defaults)

%%% ToC (table of contents) APPEARANCE
\usepackage[nottoc,notlof,notlot]{tocbibind} % Put the bibliography in the ToC
\usepackage[titles,subfigure]{tocloft} % Alter the style of the Table of Contents
\renewcommand{\cftsecfont}{\rmfamily\mdseries\upshape}
\renewcommand{\cftsecpagefont}{\rmfamily\mdseries\upshape} % No bold!

%%% END Article customizations

%%% The "real" document content comes below...

\title{Design Calculations}
\author{Gavin Hurlbut}
%\date{} % Activate to display a given date or no date (if empty),
         % otherwise the current date is printed 

\begin{document}
\maketitle

\section{Battery Charger Part Values}
See http://www.ti.com/lit/ds/symlink/bq24450.pdf

\[ V_{th} = 10.5V (1.75V/cell) \]
\[ V_{float} = 13.8V (2.30V/cell) \]
\[ V_{boost} = 14.7V (2.45V/cell) \]
\begin{align*}
I_{max-chg} &= 0.15C \\
&= 1.35A @ 9Ah \\
&= 3A @ 20Ah
\end{align*}
\[ V_{bat,min} = 8V \]
\[ I_{trickle} = 10mA \]
\[ V_{ref} = 2.30V \]

\begin{align*}
R_c &= \frac{V_{ref}}{50\mu A} \\
&= \frac{2.30V}{50\mu A} \\ 
&= 46k\Omega \\
&\simeq 46.4k\Omega \pm 1\% \tag{1}
\end{align*}

\begin{align*}
V_{float} &= V_{ref}\left(\frac{R_a + R_b + R_c}{R_c}\right) \\
\frac{V_{float}}{V_{ref}} &= \frac{R_a + R_b + R_c}{R_c} \\
R_c \left(\frac{V_{float}}{V_{ref}}\right) &= R_a + R_b + R_c \\
R_a + R_b &= R_c\left(\frac{V_{float}}{V_{ref}} - 1\right) \\
&= R_c\left(\frac{13.8V}{2.30V} - 1\right) \\
&= 5R_c \\
&= 5\left(46.4k\Omega\right) \\
&= 232k\Omega \tag{2}
\end{align*}

\begin{align*}
V_{boost} &= V_{ref}\left(\frac{R_a + R_b + R_c \parallel R_d}{R_c \parallel R_d}\right) \\
\left(R_c \parallel R_d\right)\left(\frac{V_{boost}}{V_{ref}}\right) &= R_a + R_b + R_c \parallel R_d \\
\left(R_c \parallel R_d\right)\left(\frac{V_{boost}}{V_{ref}} - 1\right) &= R_a + R_b \\
R_c \parallel R_d &= \frac{R_a + R_b}{\frac{V_{boost}}{V_{ref}} - 1} \\
&= \frac{232k\Omega}{\frac{14.7V}{2.3V} - 1} \\
&= 43.03k\Omega \\
\frac{1}{\frac{1}{R_c} + \frac{1}{R_d}} &= 43.03k\Omega \\
\frac{1}{R_c} + \frac{1}{R_d} &= \frac{1}{43.03k\Omega} \\
\frac{1}{R_d} &= \frac{1}{43.03k\Omega} - \frac{1}{R_c} \\
R_d &= \frac{1}{\frac{1}{43.03k\Omega} - \frac{1}{R_c}} \\
&= \frac{1}{\frac{1}{43.03k\Omega} - \frac{1}{46.4k\Omega}} \\
&= 592.46k\Omega \\
&\simeq 590k\Omega \pm 1\% \tag{3} \\
\\
R_c \parallel R_d &= 46.4k\Omega \parallel 590k\Omega \text{ (using nearest resistor values)} \\
&= 43.017k\Omega \tag{4}
\end{align*}

\begin{align*}
V_{th} &= V_{ref} \left(\frac{R_a + R_b +  R_c \parallel R_d}{R_b + R_c \parallel R_d}\right) \\
\left(R_b + R_c \parallel R_d\right) \left(\frac{V_{th}}{V_{ref}}\right) &= R_a + R_b + R_c \parallel R_d \\
R_b \left(\frac{V_{th}}{V_{ref}}\right) &= R_a + R_b + \left(R_c \parallel R_d\right) \left(1 - \frac{V_{th}}{V_{ref}}\right) \\
R_b &= \left(\frac{V_{ref}}{V_{th}}\right) \left(R_a + R_b + \left(R_c \parallel R_d\right) \left(1 - \frac{V_{th}}{V_{ref}}\right)\right) \\
&= \left(\frac{2.30V}{10.5V}\right) \left(232k\Omega + \left(43.017k\Omega\right) \left(1 - \frac{10.5V}{2.30V}\right)\right) \\
&= 17.225k\Omega \\
&\simeq 17.4k\Omega \pm 1\% \tag{5} \\
\\
R_a &= 232k\Omega - 17.4k\Omega \\
&= 214.6k\Omega \\
&\simeq 215k\Omega \pm 1\% \tag{6}
\end{align*}

\begin{align*}
I_{pre} = I_{trickle} &= \frac{V_{in} - V_{pre} - V_{Dext} - V_{bat,min}}{R_t} \\
R_t &=  \frac{V_{in} - V_{pre} - V_{Dext} - V_{bat,min}}{I_{trickle}} \\
&= \frac{18V - 2V - 0.7V - 8V}{10mA} \\
&= 730\Omega \\
&\simeq 732\Omega \pm 1\% \tag{7}
\end{align*}

\begin{align*}
I_{max-chg} &= \frac{V_{ilim}}{R_{sens}} \\
R_{sens} &= \frac{V_{ilim}}{I_{max-chg}} \\
&= \frac{250mV}{1.35A} \text{ (@ 9Ah)} \\
&= 185m\Omega \\
&\simeq 183m\Omega \pm 1\% \tag{8} \\
\\
R_{sense} &= \frac{250mV}{3A} \text{ (@ 20Ah)} \\
&= 83.3m\Omega \\
&\simeq 84.5m\Omega \pm 1\% \tag{9}
\end{align*}

\begin{align*}
R_p &= \frac{V_{in,min} - 0.7V}{I_{max-chg}} \times h_{fe1,min} \times h_{fe2,min} \\
&= \frac{18V - 0.7V}{1.35A} \times 150 \times 180 \text{ (@ 9Ah)} \\
&= 346k\Omega \\
&\simeq 348k\Omega \pm 1\% \tag{10} \\
\\
R_p &= \frac{18V - 0.7V}{3A} \times 150 \times 180 \text{ (@ 20Ah)} \\
&= 115.7k\Omega \\
&\simeq 118k\Omega \pm 1\% \tag{11}
\end{align*}

\begin{align*}
P_d &= I_{max-chg}\left(\frac{V_{in,max} - 0.7V}{h_{fe1,max} \times h_{fe2,max}}\right) - R_p\left(\frac{I_{max-chg}}{h_{fe1,max}\times h_{fe2,max}}\right)^2 \\
&= 1.35A\left(\frac{18V - 0.7V}{400 \times 380}\right) - 348k\Omega \left(\frac{1.35A}{400 \times 380}\right)^2 \text{ (@ 9Ah)} \\
&= 153.65\mu W - 27.45\mu W \\
&= 126.20\mu W \tag{12} \\
\\
P_d &= 3A\left(\frac{18V - 0.7V}{400 \times 380}\right) - 118k\Omega \left(\frac{3A}{400 \times 380}\right)^2 \text{ (@ 20Ah)} \\
&= 341.45\mu W - 45.97\mu W \\
&= 295.48\mu W \tag{13}
\end{align*}

\end{document}
